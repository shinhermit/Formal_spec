\documentclass{article}
\usepackage[utf8]{inputenc}
\usepackage[french]{babel}
\usepackage[T1]{fontenc}

\author{Josuah Aron}
\date{02/05/2013}
\title{TP2 de génie logiciel: spécification en Vdm++}

\begin{document}
\section{Analyse}
\subsection{Introduction: objectif du modèle}
Produire un logiciel qui permette, à tout instant, de connaître et de contrôler l'état du système:
\begin{enumerate}
\item connaître l'état:
  \begin{itemize}
  \item la liste des tâches en attente,
  \item la liste des tâches en cours d’exécution,
  \item la liste des tâches terminées,
  \item le robot qui exécute une tâche en cours d’exécution,
  \item la tâche affectée à un robot occupé.
  \item la liste des robots occupés.
  \item  \end{itemize}
\item modifier l'état
  \begin{itemize}
  \item affecter une tâche à un robot
  \item interrompre un tâche en cours
  \end{itemize}
\end{enumerate}

\subsection{Définition des besoins}
\subsection{Analyse des besoins fonctionnels}
\subsubsection{Les classes}
\begin{itemize}
\item La classe représentant l'atelier: \texttt{Shed}
\end{itemize}

\subsubsection{Les types}
\begin{itemize}
\item Le type des robots: \texttt{Robot}, correspond à la représentation interne des robots au sein de la classe \texttt{Shed}.
\item Le type des tâches: \texttt{Task}
\end{itemize}

\subsubsection{Les opérations}
Opérations exigées dans le cahier des charges:
\begin{itemize}
\item \texttt{waitingTasks}: fournit l'ensemble des tâches en attente
\item \texttt{runningTasks}: fournit l'ensemble des tâches en cours d'exécution
\item \texttt{finishedTasks}: fournit l'ensemble des tâches terminées
\item \texttt{runner}: fournit le robot qui exécute une tâche
\item \texttt{taskOf}: fournit la tâche affectée à nu robot
\item \texttt{busyRobots}: fournit l'ensemble des robots occupés
\end{itemize}
Opérations nécessaires au contrôle de l'état du système:
\begin{itemize}
\item \texttt{assignTask}: signifier l'affectation d'une tâche à un robot. Cette opération est fournie pour l'ordonnanceur.
\item \texttt{endTask}: signifier la fin de l'exécution d'un tâche par un robot. Cette opération est fournie pour les robots.
\end{itemize}

\subsection{Prévisualisation UML}
 ----------------------
| Shed                 |
 ----------------------
| robots: set of Robot |
| tasks: set of Task   |
 --------------------- 

\section{Modélisation}
\subsection{Squelette: UML -> Vdm++}
\subsection{Développement du modèle: invariants, opérations}
\subsection{Mise à jour du modèle UML}
\subsection{Modèle complet}

\section{Validation du modèle}
\subsection{Validation statique: contrôle d'intégrité}
\subsection{Validation dynamique: interpréteur}
\subsubsection{Instruction de test pour la classe XXX}

\section{Génération du code Java}
\end{document}
