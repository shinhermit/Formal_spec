\documentclass{article}
\usepackage[utf8]{inputenc}
\usepackage[french]{babel}
\usepackage[T1]{fontenc}
\usepackage{amsmath}
\usepackage{amssymb}
\usepackage{amsfonts}

\author{Josuah Aron}
\date{28/04/2013}
\title{TP1 génie logigicel: interprétation et vérification d'une spécification}

\begin{document}
\maketitle
\tableofcontents

\section{Introduction}
Dans les obligations de preuve, Pré et Post désignent les précondition et postcondition de la spécification (et non de l'interprétation) de l'opération considérée, appliquées aux opérations du module (implémentations).\\
Lorsque cela est utile, nous suffixons Pré et Post par le nom de l'opération, afin de distinguer les différentes formules citées. Par exemple, \texttt{Pré\_calcul} désigne la précondition de l'opération \texttt{calcul}.

\section{Le module Article00}
La spécification d'architecture:\\
{\tt
F\_ARTICLE00: String * Rat * Nat * Int $\rightarrow$ Article00\\
ARTICLE00 = F\_ARTICLE00[STRING][RAT][NAT][INT]\\
}
indique que:
\begin{enumerate}
\item le module de première classe \texttt{F\_ARTICLE00}\footnote{F comme \og First class\ \fg} fournira un implémentation de la spécification \texttt{Article00} à condition qu'on lui fournisse les implémentations des spécification String, Rat, Nat et Int (cf. ref \cite{Architectural:specifications:in:CASL});
\item le module \texttt{ARTICLE00} implémente la spécification \texttt{Article00} par instanciation de la première classe \texttt{F\_ARTICLE00} avec les modules \texttt{STRING, RAT, NAT} et \texttt{INT}.
\end{enumerate}

\subsection{Interprétation de la sort Article}
\subsubsection{La structure de donnée Article}
\begin{verbatim}
struct Article
  ref: String
  prix: Rat
  stock: Int
  delai: Nat
  taux: Nat
\end{verbatim}
{\tt
\underline{Invariant:}
$$ prix \ge 0 \wedge stock \ge 0 $$
}

\subsection{L'opération creer\_article}
\subsubsection{Interprétation}
{\tt
creer\_article(ref:String, prix:Rat, delai:Nat; taux:Nat; stock:Int)  article : struct Article\\
\underline{Pré}
$$ prix \ge 0 \wedge stock \ge 0$$
\underline{Post}
$$ article.ref = ref \wedge article.prix = prix \wedge article.stock = stock $$
$$  \wedge\ article.delai = delai \wedge article.taux = taux $$
}

\subsubsection{Spécification}
{\tt
creer\_article(ref:String, prix:Rat, delai:Nat; taux:Nat; stock:Int)  article : struct Article\\
\underline{Pré(ref, prix, delai, taux, stock)}
$$ prix \ge 0 \wedge stock \ge 0$$
\underline{Post(ref, prix, delai, taux, stock, article)}
$$ reference(article) = ref \wedge prix(article) = prix \wedge delai\_livraison(article) = delai$$
$$ \wedge\ taux\_promotion(article) = taux \wedge stock(article) = stock$$
}

\subsubsection{Obligation de preuve}
$$ \forall ref:String, prix:Rat, delai,taux:Nat, stock:Int, $$
$$ Pre(ref, prix, delai, taux, stock) \implies creer\_article(ref, prix, delai, taux, stock): struct\ Article $$
$$\wedge\ Post(ref, prix, delai, taux, stock, article) $$

L'expression $creer\_article(ref, prix, delai, taux, stock): struct\ Article$ signifie que le résultat de l'opération préserve l'invariant du type.

\subsection{L'opération \_\_est\_en\_rupture\_de\_stock}
\subsubsection{Interprétation}
{\tt
\_\_est\_en\_rupture\_de\_stock(article:Article)  rupture:booléen\\
\underline{Pré(article)}
$$ true $$
\underline{Post(article, rupture)}
$$ rupture \Leftrightarrow article.stock = 0$$
}

\subsubsection{Spécification}
{\tt
\_\_est\_en\_rupture\_de\_stock(article:Article)  rupture:booléen\\
\underline{Pré(article)}
$$ true$$
\underline{Post(article, rupture)}
$$ rupture \Leftrightarrow stock(article) = 0$$
}

\subsubsection{Obligation de preuve}
$$ \forall\ art:Article,\ rupture:booléen,$$
$$ rupture = art\ est\_en\_rupture\_de\_stock \implies Post(art, rupture) $$

\subsection{L'opération reference}
\subsubsection{Interprétation}
{\tt
reference(article:Article)  ref:String\\
\underline{Pré(article)}
$$ true $$
\underline{Post(article, ref)}
$$ ref = article.ref $$
}

\subsubsection{Spécification}
{\tt
reference(art:Article)  ref:String\\
\underline{Pré(art)}
$$ true $$
\underline{Post(art, ref)}
$$ art = creer\_article(\text{ref},\ prix(art),\ delai\_livraison(art),\ taux\_promotion(art),\ stock(art)) $$
}

\subsubsection{Obligation de preuve}
$$ \forall\ art:Article,\ ref:String,\quad ref = reference(art) \implies Post(art,ref) $$

\subsection{L'opération prix}
\subsubsection{Interprétation}
{\tt
prix(article:Article)  lePrix:Rat\\
\underline{Pré(art)}
$$ true $$
\underline{Post(article, lePrix)}
$$ lePrix = article.prix $$
}

\subsubsection{Spécification}
{\tt
prix(art:Article)  lePrix:Rat\\
\underline{Pré(art)}
$$ true $$
\underline{Post(art, lePrix)}
$$ art = creer\_article(reference(art),\ \text{lePrix},\ delai\_livraison(art),\ taux\_promotion(art),\ stock(art)) $$
}

\subsubsection{Obligation de preuve}
$$ \forall\ art:Article,\ lePrix:Rat,\quad lePrix = prix(art) \implies Post(art,lePrix) $$

\subsection{L'opération stock}
\subsubsection{Interprétation}
{\tt
stock(article:Article) leStock:Int\\
\underline{Pré(article)}
$$ true $$
\underline{Post(article, leStock)}
$$ stock = article.stock $$
}

\subsubsection{Spécification}
{\tt
stock(art:Article) leStock:Int\\
\underline{Pré(art)}
$$ true $$
\underline{Post(art, leStock)}
$$ art = creer\_article(reference(art),\ prix(art),\ delai\_livraison(art),\ taux\_promotion(art),\ \text{leStock}) $$
}

\subsubsection{Obligation de preuve}
$$ \forall\ art:Article,\ leStock:Int,\quad leStock = stock(art) \implies Post(art, leStock)$$

\subsection{L'opération delai\_livraison}
\subsubsection{Interprétation}
{\tt
delai\_livraison(article:Article) delai:Nat\\
\underline{Pré(article)}
$$ true $$
\underline{Post(article, delai)}
$$ delai = article.delai $$
}

\subsubsection{Spécification}
{\tt
delai\_livraison(article:Article) delai:Nat\\
\underline{Pré(art)}
$$ true $$
\underline{Post(art, delai)}
$$ art = creer\_article(reference(art),\ prix(art),\ \text{delai},\ taux\_promotion(art),\ stock(art)) $$
}

\subsubsection{Obligation de preuve}
$$ \forall\ art:Article,\ delai:Nat,\quad delai = delai\_livraison(art) \implies Post(art, delai) $$

\subsection{L'opération taux\_promotion}
\subsubsection{Interprétation}
{\tt
taux\_promotion(article:Article) taux:Nat\\
\underline{Pré(article)}
$$ true $$
\underline{Post(article, taux)}
$$ taux = article.taux $$
}

\subsubsection{Spécification}
{\tt
taux\_promotion(art:Article) taux:Nat\\
\underline{Pré(art)}
$$ true $$
\underline{Post(art, taux)}
$$ art = creer\_article(reference(art),\ prix(art),\ delai\_livraison(art),\ \text{taux},\ stock(art)) $$
}

\subsubsection{Obligation de preuve}
$$ \forall\ art:Article,\ taux:Nat,\quad taux = taux\_promotion(art) \implies Post(art, taux) $$

\subsection{L'opération prix\_promotion}
\subsubsection{Interprétation}
{\tt
prix\_promotion(article:Article) promo:Rat\\
\underline{Pré(article)}
$$ true $$
\underline{Post(article, promo)}
$$ promo = article.prix - (article.prix * article.taux) / 100 $$
}

\subsubsection{Spécification}
{\tt
prix\_promotion(art:Article) promo:Rat\\
\underline{Pré(art)}
$$ true $$
\underline{Post(art, promo)}
$$ promo = prix(art) - ( prix(art) * taux\_promotion(art) )/100$$
}

\subsubsection{Obligation de preuve}
$$ \forall\ art:Article,\ promo:Rat,\quad promo = prix\_promotion(art) \implies Post(art, promo) $$

\section{Le module Article}
La spécification d'architecture:\\
{\tt
ARTICLE: Article given ARTICLE00\\
}
indique que le module \texttt{ARTICLE} implémente la spécification \texttt{Article} en utilisant le module \texttt{ARTICLE00} (cf. ref \cite{Architectural:specifications:in:CASL}). 

\subsection{L'opération approvisionner}
\subsubsection{Interprétation}
{\tt
approvisionner(art1:Article, quant:Int) art2:Article\\
\underline{Pré(art1, quant)}
$$ quant \ge 0\ \wedge\ art1.stock = 0$$
\underline{Post(art1, quant, art2)}
$$ art2.stock = \text{quant}\ \wedge\ art2.ref = art1.ref\ \wedge\ art2.prix = art1.prix$$
$$ \ \wedge\ art2.delai = art1.delai\ \wedge\  art2.taux = art1.taux$$
}

\subsubsection{Spécification}
{\tt
approvisionner(art1:Article, quant:Int) art2:Article\\
\underline{Pré(art1, quant)}
$$ quant \ge 0\ \wedge\ art est\_en\_rupture\_de\_stock$$
\underline{Post(art1, quant, art2)}
$$ stock(art2) = \text{quant}\ \wedge\ reference(art2) = reference(art2)\ \wedge\ prix(art1) = prix(art2) $$
$$ \ \wedge\ taux\_promotion(art2) = taux\_promotion(art1)\ \wedge\ prix\_promotion(art2) = prix\_promotion(art2)$$
$$ \ \wedge\ delai\_livraison(art2) = delai\_livraison(art1)$$
}

\subsubsection{Obligation de preuve}
$$ \forall\ art1,art2:Article, quant:Int,$$
$$ Pre(art1,quant)\ \wedge\ art2 = approvisionner(art1,quant) \implies\ Post(art1, quant, art2) $$

\subsection{L'opération promotionner}
\subsubsection{Interprétation}
{\tt
promotionner(art1:Article, taux:Nat) art2:Article\\
\underline{Pré(art1, taux)}
$$ true $$
\underline{Post(art1, taux, art2)}
$$ art2.taux = \text{taux}\ \wedge\ art2.ref = art1.ref\ \wedge\ art2.prix = art1.prix$$
$$ \ \wedge\ art2.delai = art1.delai\ \wedge\  art2.stock = art1.stock$$
}

\subsubsection{Spécification}
{\tt
promotionner(art1:Article, taux:Nat) art2:Article\\
\underline{Pré(art1, taux)}
$$ true $$
\underline{Post(art1, taux, art2)}
$$ taux\_promotion(art2) = \text{taux}\ \wedge\ reference(art2) = reference(art2)\ \wedge\ prix(art1) = prix(art2) $$
$$ \ \wedge\ stock(art2) = stock(art1)\ \wedge\ prix\_promotion(art2) = prix\_promotion(art2)$$
$$ \ \wedge\ delai\_livraison(art2) = delai\_livraison(art1)$$
}

\subsubsection{Obligation de preuve}
$$ \forall\ art1,art2:Article, taux:Nat,$$
$$ art2 = promotionner(art1,taux) \implies\ Post(art1, taux, art2) $$

\subsection{L'opération livrer}
\subsubsection{Interprétation}
{\tt
livrer(art1:Article, quant:Int) art2:Article\\
\underline{Pré(art1, quant)}
$$ quant \ge 0\ \wedge\ quant \le art1.stock $$
\underline{Post(art1, quant, art2)}
$$ art2.stock = art1.stock - \text{taux}\ \wedge\ art2.ref = art1.ref\ \wedge\ art2.prix = art1.prix$$
$$ \ \wedge\ art2.delai = art1.delai\ \wedge\  art2.stock = art1.stock$$
}

\subsubsection{Spécification}
{\tt
livrer(art1:Article, quant:Int) art2:Article\\
\underline{Pré(art1, quant)}
$$ quant \ge 0\ \wedge\ quant \le stock(art1) $$
\underline{Post(art1, quant, art2)}
$$ stock(art2) = stock(art1) - \text{quant}\ \wedge\ reference(art2) = reference(art2)\ \wedge\ prix(art1) = prix(art2) $$
$$ \ \wedge\ taux\_promotion(art2) = taux\_promotion(art1)\ \wedge\ prix\_promotion(art2) = prix\_promotion(art2)$$
$$ \ \wedge\ delai\_livraison(art2) = delai\_livraison(art1)$$
}

\subsubsection{Obligation de preuve}
$$ \forall\ art1,art2:Article, quant:Int,$$
$$ Pre(art1,quant)\ \wedge\ art2 = livrer(art1,quant) \implies\ Post(art1, quant, art2) $$

\section{Le module Gestionnaire00}
La spécification d'architecture:\\
{\tt
NAT: Nat\\
SET: Set[sort elem] given NAT\\
GESTIONNAIRE00: Gestionnaire00[sort S] given SET\\
}
indique que:
\begin{enumerate}
\item le module \texttt{NAT} implémente la spécification \texttt{Nat}
\item le module \texttt{SET} implémente la spécification générique Set en utilisant le module \texttt{NAT}
\item le module \texttt{GESTIONNAIRE00} implémente la spécification générique \texttt{Gestionnaire00[sort S]} en utilisant le module \texttt{SET}
\end{enumerate}

\subsection{Interprétation de la sort Gestionnaire}
\subsubsection{La structure de donées générique Gestionnaire}
\begin{verbatim}
struct Gestionnaire[elem]
  elems: Set[elem]
  ref: String
\end{verbatim}

\subsection{L'opération vide}
\subsubsection{Interprétation}
{\tt
vide() gest:Gestionnaire\\
\underline{Pré}
$$ true $$
\underline{Post(gest)}
$$ gest.elems = \{\} $$
}

\subsubsection{Spécification}
{\tt
vide() gest:Gestionnaire\\
\underline{Pré}
$$ true $$
\underline{Post(gest)}
$$ gest = \{\} $$
}

\subsubsection{Obligation de preuve}
$$ \forall\ gest:Gestionnaire,\quad gest = vide \implies gest = \{\} $$

\subsection{L'opération inserer}
\subsubsection{Interprétation}
{\tt
inserer(gest1:Gestionnaire[Elem], elem:Elem) gest2:Gestionnaire[Elem]\\
\underline{Pré(gest1, elem)}
$$ true $$
\underline{Post(gest1, elem, gest2)}
$$ gest2.elems = gest2.elems + elem\ \wedge\ gest2.ref = gest1.ref $$
}

\subsubsection{Spécification}
{\tt
inserer(gest1:Gestionnaire[Elem], elem:Elem) gest2:Gestionnaire[Elem]\\
\underline{Pré(gest1, elem)}
$$ true $$
\underline{Post(gest1, elem, gest2)}
$$ gest2 = gest1 + elem $$
où + est l'opeération de Set[sort elem]
}

\subsubsection{Obligation de preuve}
$$ \forall\ gest1,gest2:Gestionnaire[Elem],\ elem:Elem,$$
$$gest2 = inserer(gest1, elem) \implies Post(gest1, elem, gest2) $$

\subsection{L'opération supprimer}
\subsubsection{Interprétation}
{\tt
supprimer(gest1:Gestionnaire[Elem], elem:Elem) gest2:Gestionnaire[Elem]\\
\underline{Pré}
$$ Pre\_supprimer(gest1,elem) \Leftrightarrow Pre\_\_-\_\_(gest1.elems,elem) $$
\underline{Post}
$$ Post\_supprimer(gest1,elem) \Leftrightarrow Post\_\_-\_\_(gest1.elems,elem,gest2.elems) $$
}

\subsubsection{Spécification}
{\tt
supprimer(gest1:Gestionnaire[Elem], elem:Elem) gest2:Gestionnaire[Elem]\\
\underline{Pré}
$$ Pre\_supprimer(gest1,elem) \Leftrightarrow Pre\_\_-\_\_(gest1,elem) $$
\underline{Post}
$$ Post\_supprimer(gest1,elem,gest2) \Leftrightarrow Post\_\_-\_\_(gest1,elem,gest2) $$
}

\subsubsection{Obligation de preuve}
$$ \forall\ gest1,gest2:Gestionnaire[Elem],\ elem:Elem,$$
$$ Pre(gest1,elem)\ \wedge\ gest2 = supprimer(gest1,elem) \implies Post(gest1,elem,gest2)$$

\section{Le module Gestionnaire}
La spécification d'architecture:\\
{\tt
NAT: Nat
STRING : String given NAT
F\_GESTIONNAIRE: Gestionnaire00[sort S] * String $\rightarrow$ Gestionnaire[sort S]\\
GESTIONNAIRE = F\_GESTIONNAIRE[GESTIONNAIRE00][STRING]\\
}
indique que:
\begin{enumerate}
\item le module \texttt{NAT} impleménte la spécification \texttt{Nat}
\item le module \texttt{STRING} implémente la spécification \texttt{String} en utilisant le module NAT
\item le module de première classe \texttt{F\_GESTIONNAIRE} fournira une implémentation de la spécification générique \texttt{Gestionnaire} à condition qu'on lui fournisse une implémentation des spécifications \texttt{Gestionnaire00} et \texttt{String};
\item le module \texttt{GESTIONNAIRE} implémente la spécification \texttt{Gestionnaire} par instanciation de la première classe \texttt{F\_GESTIONNAIRE}, avec les modules \texttt{GESTIONNAIRES00} et \texttt{STRING}.
\end{enumerate}

\subsection{L'opération gestionnaire}
\subsubsection{Interprétation}
{\tt
gestionnaire\_vide(ref:String) gest:Gestionnaire[S]\\
\underline{Pré(ref)}
$$ true $$
\underline{Post(ref, gest)}
$$ Post\_vide\ \wedge\ gest.ref = ref$$
}

\subsubsection{Spécification}
{\tt
gestionnaire\_vide(ref:String) gest:Gestionnaire[S]\\
\underline{Pré(ref)}
$$ true $$
\underline{Post(ref, gest)}
$$ Post\_vide\ \wedge\ reference(gest) = ref$$
}

\subsubsection{Obligation de preuve}
$$ \forall\ ref:String, gest:Gestionnaire,\quad gest = gestionnaire\_vide(ref) \implies Post(ref, gest)$$

\subsection{L'opération reference}
\subsubsection{Interprétation}
{\tt
reference(gest:Gestionnaire[S]) ref:String\\
\underline{Pré(gest)}
$$ true $$
\underline{Post(gest, ref)}
$$ ref = gest.ref $$
}

\subsubsection{Spécification}
{\tt
reference(gest:Gestionnaire[S]) ref:String\\
\underline{Pré(gest)}
$$ true $$
\underline{Post(gest, ref)}
$$ gest = gestionnaire\_vide(\text{ref}) $$
}

\subsubsection{Obligation de preuve}
$$ \forall gest:Gestionnaire[Elem], ref:String, \quad ref = reference(gest) \implies Post(gest, ref) $$

\section{Le module Catalogue}
La spécification d'architecture:\\
{\tt
F\_CATALOGUE:  Article * Gestionnaire[sort Article fit sort S |$\rightarrow$ Article] $\rightarrow$ Catalogue\\
GESTIONNAIRE\_ARTICLE: Gestionnaire[sort Article fit sort S |$\rightarrow$ Article] given ARTICLE\\
CATALOGUE = F\_CATALOGUE[ARTICLE][GESTIONNAIRE\_ARTICLE]\\
}
indique que:
\begin{enumerate}
\item le module de première classe \texttt{F\_CATALOGUE} fournira une implémentation de la spécification \texttt{Catalogue} à condition qu'on lui fournisse les implémentations des spécifications \texttt{Article} et \texttt{Gestionnaire[sort Article]};
\item le module intermédiaire \texttt{GESTIONNAIRE\_ARTICLE} implémente la spécification \texttt{Gestionnaire[sort Article]};
\item le module \texttt{CATALOGUE} implémente la spécification \texttt{Catalogue} par instanciation de la première classe \texttt{F\_CATALOGUE} avec les modules \texttt{ARTICLE} et \texttt{GESTIONNAIRE\_ARTICLE}.
\end{enumerate}

\subsection{Interprétation de la sort Catalogue}
\subsubsection{La spécialisation Catalogue}
\begin{verbatim}
struct Catalogue = Gestionnaire[struct Article]
\end{verbatim}

\subsection{L'opération catalogue\_vide}
\subsubsection{Interprétation}
{\tt
catalogue\_vide(ref:String) cat:Catalogue\\
\underline{Pré}
$$ Pré\_catalogue\_vide\ \Leftrightarrow\ Pré\_gestionnaire\_vide $$
\underline{Post}
$$ Post\_catalogue\_vide \Leftrightarrow\ Prost\_gestionnaire\_vide $$
}

\subsubsection{Spécification}
{\tt
catalogue\_vide(ref:String) cat:Catalogue\\
\underline{Pré}
$$ Pré\_catalogue\_vide\ \Leftrightarrow\ Pré\_gestionnaire\_vide $$
\underline{Post}
$$ Post\_catalogue\_vide \Leftrightarrow\ Prost\_gestionnaire\_vide $$
}

\subsubsection{Obligation de preuve}
Obligation de preuve de \texttt{gestionnaire\_vide}.

\section{Le module Collection}
La spécification d'architecture:\\
{\tt
F\_COLLECTION:  Catalogue * Gestionnaire[sort Catalogue fit sort S |$\rightarrow$ Catalogue] $\rightarrow$ Collection\\
GESTIONNAIRE\_CATALOGUE: Gestionnaire[sort Catalogue fit sort S |$\rightarrow$ Catalogue] given CATALOGUE\\
COLLECTION = F\_COLLECTION[CATALOGUE][GESTIONNAIRE\_CATALOGUE]\\
}
indique que:
\begin{enumerate}
\item le module de première classe \texttt{F\_COLLECTION} fournira une implémentation de la spécification \texttt{Catalogue} à condition qu'on lui fournisse les implémentations des spécifications \texttt{Catalogue} et \texttt{Gestionnaire[sort Catalogue]};
\item le module intermédiaire \texttt{GESTIONNAIRE\_CATALOGUE} implémente la spécification \texttt{Gestionnaire[sort Catalogue]};
\item le module \texttt{COLLECTION} implémente la spécification \texttt{Collection} par instanciation de la première classe \texttt{F\_COLLECTION} avec les modules \texttt{CATALOGUE} et \texttt{GESTIONNAIRE\_CATALOGUE}.
\end{enumerate}

\subsection{Interprétation de la sort Collection}
\subsubsection{La spécialisation Collection}
\begin{verbatim}
struct Collection = Gestionnaire[struct Catalogue]
\end{verbatim}

\subsubsection{Interprétation}
{\tt
collection\_vide(ref:String) cat:Collection\\
\underline{Pré}
$$ Pré\_collection\_vide\ \Leftrightarrow\ Pré\_gestionnaire\_vide $$
\underline{Post}
$$ Post\_collection\_vide \Leftrightarrow\ Prost\_gestionnaire\_vide $$
}

\subsubsection{Spécification}
{\tt
collection\_vide(ref:String) cat:Collection\\
\underline{Pré}
$$ Pré\_collection\_vide\ \Leftrightarrow\ Pré\_gestionnaire\_vide $$
\underline{Post}
$$ Post\_collection\_vide \Leftrightarrow\ Prost\_gestionnaire\_vide $$
}

\subsubsection{Obligation de preuve}
Obligation de preuve de \texttt{gestionnaire\_vide}.

\section{Le module Systeme}
La spécification d'architecture:\\
{\tt
F\_SYSTEME: Collection * String $\rightarrow$ Systeme\\
SYSTEME\_VENTE = F\_SYSTEME[COLLECTION][STRING]\\
}
indique que:
\begin{enumerate}
\item le module de première classe \texttt{F\_SYSTEME} fournira une implémentation de la spécification \texttt{Systeme} à condition qu'on lui fournisse les implémentations des spécifications \texttt{Collection} et \texttt{String};
\item le module \texttt{SYSTEME\_VENTE} fournit une implémentation de la spécification \texttt{Systeme} par instanciation de la première classe \texttt{F\_SYSTEME} avec les modules \texttt{COLLECTION} ET \texttt{STRING}.
\end{enumerate}

\subsection{Interprétation de la sort Systeme}
\subsubsection{La structure de donnée Systeme}
\begin{verbatim}
struct Systeme
  ref: String
  coll: struct Collection
\end{verbatim}

\subsection{L'opération Systeme\_vide}
\subsubsection{Interprétation}
{\tt
system\_vide(ref:String) sys:Systeme\\
\underline{Pré(ref)}
$$ true $$
\underline{Post(ref, sys)}
$$ sys.ref = ref\ \wedge\ sys.coll = vide $$
}

\subsubsection{Spécification}
{\tt
system\_vide(ref:String) sys:Systeme\\
\underline{Pré(ref)}
$$ true $$
\underline{Post(ref, sys)}
$$ reference(sys) = ref\ \wedge\ collection\_de\ sys = vide $$
}

\subsubsection{Obligation de preuve}
$$ \forall sys:Systeme,\ ref:String,\quad sys = systeme\_vide(ref) \implies Post(ref, sys) $$

\subsection{L'opération changer\_collection}
\subsubsection{Interprétation}
{\tt
changer\_collection(sys1:Systeme, coll:Collection) syst2:Systeme\\
\underline{Pré(sys1, coll)}
$$ true $$
\underline{Post(sys1, coll, sys2)}
$$ sys2.ref = sys1.ref\ \wedge\ sys2.coll = coll $$
}

\subsubsection{Spécification}
{\tt
changer\_collection(sys1:Systeme, coll:Collection) syst2:Systeme\\
\underline{Pré(sys1, coll)}
$$ true $$
\underline{Post(sys1, coll, sys2)}
$$ reference(sys2) = reference(sys)\ \wedge\ collection\_de\ sys2 = coll $$
}

\subsubsection{Obligation de preuve}
$$ \forall sys1,sys2:Systeme,\ coll:Collection,$$
$$ sys2 = changer\_collection(sys1, coll) \implies Post(sys1, coll, sys2)$$

\subsection{L'opération reference}
\subsubsection{Interprétation}
{\tt
reference(sys:Systeme) ref:String\\
\underline{Pré(sys)}
$$ true $$
\underline{Post(sys, ref)}
$$ ref = sys.ref $$
}

\subsubsection{Spécification}
{\tt
reference(sys:Systeme) ref:String\\
\underline{Pré(sys)}
$$ true $$
\underline{Post(sys, ref)}
$$ sys = changer\_collection(systeme\_vide(ref), collection\_de\ \text{sys}) $$
}

\subsubsection{Obligation de preuve}
$$ \forall sys:Systeme,\ ref:String,\quad ref = reference(sys) \implies Post(sys, ref) $$

\subsection{L'opération collection\_de\_\_}
\subsubsection{Interprétation}
{\tt
collection\_de\_\_(sys:Systeme) coll:Collection\\
\underline{Pré(sys)}
$$ true $$
\underline{Post(sys, coll)}
$$ coll = sys.coll $$
}

\subsubsection{Spécification}
{\tt
collection\_de\_\_(sys:Systeme) coll:Collection\\
\underline{Pré(sys)}
$$ true $$
\underline{Post(sys, coll)}
$$ sys = changer\_collection(systeme\_vide(reference(sys)), \text{coll}) $$
}

\subsubsection{Obligation de preuve}
$$ \forall\ sys:Systeme,\ coll:Collection,\quad coll = collection\_de\ sys \implies Post(sys, coll)$$

\bibliographystyle{plain}
\bibliography{interpretation-verification.bib}
\end{document}
